\setcounter{chapter}{18}
\chapter{Dynamic Data Structures}
Structures are used to represent objects that are represented by multiple basic data types.

\section{Structures}
\begin{verbatim}
    struct flightType{
        char ID[7];
        int altitude;
    }
\end{verbatim}

We can declare a variables using, 
\begin{verbatim}
    struct flightType plane;
\end{verbatim}
To access individual elements we can do,
\begin{verbatim}
    plane.ID = "test";
    plane.altitude = 100;
\end{verbatim}

\subsection{typedef}
We can use typedef to define our own aggregate types,
\begin{verbatim}
    typedef struct flightType Flight;
\end{verbatim}

\subsection{Array of structure}
\begin{verbatim}
    Flight aircraft[100];
\end{verbatim}
We have an array of Flight which can store 100 Flights.

We can access using, 
\begin{verbatim}
    aircraft[0].heading
\end{verbatim}

We can also create pointers to structures, 
\begin{verbatim}
    Flight *airPtr;
    airPtr = &aircraft[30];
\end{verbatim}
to access any field we can do,
\begin{verbatim}
    (*airPtr).longitude;
    airPtr->longitude
\end{verbatim}



\section{Dynamic Memory Allocation}
Fixed data has problems, (1) We can't go over. (2). Adding, removing etc can take a toll.

Memory objects in C are allocated to,
\begin{enumerate}
    \item run-time stack: variables declared local to functions.
    \item global data section: global variables.
    \item heap: dynamically allocated data objects.
\end{enumerate}

At high level, a memory allocated (malloc) manages an area of memory called the heap. A program can make a request to malloc for blocks of memory of a particular size. malloc then locates a block of the size and reserves it by marking it as allocated and returns a pointer to the block. Once a block is allocated it will stay allocated unless we deallocate it by calling "free" in C unlike the stack where after the function call it is technically deallocated.

\subsubsection{Dynamically Sized Arrays}
Dynamic allocation and reallocation are handled by malloc and free.
\begin{verbatim}
    int numAircraft; 
    Flight *planes;

    planes = (Flight *) malloc(24 * numAircraft);
\end{verbatim}

malloc allocates a contiguous region of memory on the heap of the size in bytes.  If the heap has enough unclaimed memory, malloc returns a pointer to the allocated region. In this case we have $24$ because one Flight takes up 24 bytes of memory (depends on the int, char, double etc in the struct).

malloc returns NULL if the current allocation cannot be accomplished. Because malloc needs to be compatible with any data type we need to type cast it depending on what we're using it for.

\begin{verbatim}
    int num;
    Flight *planes;

    planes = (Flight *) malloc(sizeof(Flight) * num);
    if (planes == NULL) {
        printf("Error in malloc");
    }
\end{verbatim}





 
