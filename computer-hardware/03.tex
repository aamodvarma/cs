\chapter{IO}

\section{Privilege, Priority and Memory Address Space}
\subsection{Privilege}
We could either have supervisor privilege or user privilege. Idea is that we don't want everyone to have complete control over things like HALT, some memory locations. 

\subsection{Priority}
All about urgency. Everyone program is assigned a priority from 0 to 7. Higher the priority it will be executed first. 


\subsection{Processor Status Register}
Each program executing has associated with it two important registers the PC and the PSR. The PSR stores the privilege and priority assigned to the program.

PSR[15]=1 means unprivileged and PSR[15]=0 means supervisor. Bits[10:8] specify the priority level from 0 to 7. Bits[2:0] store NZP condition codes.



\subsection{Organization of Memory}
We have privileged memory from $x0000$ to  $x2FFFF$.  They contain data structures and code of the operating system. Require supervisor privilege to access. Referred to as system space.

Locations $x3000$ to  $xFDFF$ are unprivileged memory locations. Referred to as user space.

From  $xFE00$ to $xFFFF$ do not correspond to memory at all. So the last address is $xFDFF$. But  $xFE00$ to  $xFFFF$ are used to identify registers that take part in input and output functions. For instance  $PSR$ is assigned address $xFFFC$ and the Master Control Register (MCR) is at  $xFFFE$. 


Both the supervisor stack and the user stack has a stack pointer that indicates the start of the stack. Only one of the two stacks are activate at a time. Register 6 is used as the stack pointer for active stacks. 


\section{Input/Output}

I/O devices are mapped to a set of addresses that are allocated to I/O device registers rather than memory locations. LC-3 uses memory mapped I/O. Where xFE00 to  xFFFF are reserved for input/output device registers.


\subsection{Asynchronous vs Synchronous}
An asynchronous mechanism would require some protocol or handshaking. So we need a flag to indicate whether someone has or has not typed a character or whether or not a character has been  displayed in the monitor. So every time a typist types something the ready bit is set to 1 and if the character is read by the computer it is set back to 0.

Synchronous is when the computer checks for characters at a specified interval of time. Which requires the typist to be synchronized with the computers time period.


\subsection{Interrupt Driven vs. Polling}
Interrupt-driven means the processor is interrupted and told that a key has been struck.  But polling means that the processor will check again and again if something has been struck.

\subsection{Input from Keyboard}
\subsubsection{KBDR and KBSR}
We have two main registers. KBDR is assigned xFE02 while KBSR is assigned xFE00. 

\begin{enumerate}
    \item  For KBDR, Bit[7:0] is used for the ASCII value. 
\item  For KBSR, Bit[15] (ready bit) determines whether a key has to be read and Bit[14] determines whether we want to interrupt.
\end{enumerate}


\subsubsection{Service Routine}
When a key is struct the ASCII code is loaded into KBDR[7:0] and KBSR[15] is set to 1. After reading KBSR[15] is cleared.

If polling is used, the processor repeatedly checks KBSR[15] until it is set and then can load the ASCII from KBDR.

\begin{verbatim}
    START
        LDI    R1, A
        BRzp   START
        LDI    R0, B
        BRnzp  NEXT_TASK
    A   .fill  xFE00
    B   .fill  xFE02
\end{verbatim}

Loads the ASCII value inputted into R0.

\subsubsection{Memory-Mapped Input}
Three steps, 
\begin{enumerate}
    \item MAR is loaded with the address of a device register.
    \item Memory is read, MDR is loaded with contents at the memory location.
    \item DR is loaded with MDR contents.
\end{enumerate}

\subsection{Output to Monitor}
\subsubsection{DDR and DSR}
We have two main registers. DDR is assigned xFE06 while DSR is assigned xFE04. 

\begin{enumerate}
    \item  For DDR, Bit[7:0] is used for the ASCII value to be shown
\item  For DSR, Bit[15] (ready bit) determines whether a key has to be read and Bit[14] determines whether we want to interrupt.
\end{enumerate}

\subsubsection{Service routine}
When a key is to be shown the ASCII code is loaded into DDR[7:0] and DSR[15] is set to 1. After being outputted by the monitor DSR[15] is cleared.

If polling is used, the processor repeatedly checks DSR[15] until it is set and then can output the ASCII.

\begin{verbatim}
    START
        LDI    R1, A
        BRzp   START
        STI    R0, B
        BRnzp  NEXT_TASK
    A   .fill  xFE04
    B   .fill  xFE06
\end{verbatim}

Displays the ASCII value in R0 in the monitor.

\subsubsection{Memory-Mapped Output}
\begin{enumerate}
    \item MAR is loaded with the address of a device register.
    \item MDR is loaded with contents to be written.
    \item Memory is written resulting in the contents of MDR being stored in the address of MAR.
\end{enumerate}


\section{Trap Routines}
\subsection{Introduction}
For I/O we need to know the following, 
\begin{enumerate}
    \item Hardware data registers.
    \item Hardware status registers.
    \item Asynchronous nature of keyboard input relative to executing program.
\end{enumerate}


\subsection{Trap Mechanism}
\begin{enumerate}
    \item A set of service routines executed  on behalf of user programs. There are 256 service routines.
    \item A table of starting addresses stored from x0000 to x00FF.
    \item The TRAP instruction. The user uses the TRAP instruction when it wants the OS to execute a specific service routine.
    \item A linkage back to the program. The routine must have a mechanism to return control to the user program.
\end{enumerate}

\subsection{Trap Instruction}
The instruction works by, 
\begin{enumerate}
    \item Changing PC to starting address of relevant service routine.
    \item Providing a way to get back to the program that executed the TRAP instruction.
\end{enumerate}

Opcode is 1111 and the trap vector is Bits[7:0] which identifies the service routine the program wants the OS to execute.

The EXECUTE phase of the TRAP instruction cycle does three things, 
\begin{enumerate}
    \item PSR and PC are pushed onto stem stack. As PC is incremented the return linkage is saved in the PC. If in user mode R6 is pointing to the user stack. Before PSR and PC is pushed onto system stack, R6 is first stored in SAVED\_USP and Saved\_SSP is loaded into R6.
    \item PSR[15] is set to 0 as the service routine needs supervisor privilege.
    \item Trap vector is zero extended to form an address. So x23 is extended to the address x0023. The location x0023 contains the address of the trap handler eg. x04A0.
\end{enumerate}

\subsection{RTI}
RTI means Return from Trap or Interrupt.  Opcode is 1000 with no operands. Pops the two top values of the system stack into the PC and PSR. PC will then contain the address following the TRAP instruction. If PSR[15] was in usermode the stack pointers are adjusted back how it was, so Saved\_SSP is loaded with R6 and R6 is loaded with  Saved\_USP.


\subsection{Summary}
Execution of TRAP x23 first causes PSR and incremented PC to be pushed onto system stack. Then contents of x0023 (x04A0) are loaded onto PC. 

Next cycle starts with FETCH of the contents of x04A0 which is the first instruction of the service routine. The routine is executed and then ends with RTI which loads PC and PSR with the top two elements in the system stack. 

Examples uses, 

\begin{verbatim}
    LD    R0, ASCIINewLine
    TRAP  x21
    LEA   R0, Message
    TRAP  x22
    LD    R0, ASCIINewLine
    TRAP  x21
\end{verbatim}

\subsection{GETC}
Invoke TRAP x21 and it displays the ASCII in R0 into the monitor.

\subsection{PUTS}
Invoke TRAP x22 and it displays the string starting from the address in R0 until the end (x0000) into the monitor.


\section{Interrupts}
We see the case where the I/O is controlled by the I/O device rather than processor polling. An I/O device may or may not have anything to do with the program that is currency running can, \begin{enumerate}
    \item Force stop the current program
    \item Execute a program that carries out needs of the device
    \item Resume execution of initial program
\end{enumerate}

\subsection{Why}
Polling wastes a lot of time re executing LDI and BR until the ready bit is set. With interrupt we don't need to do that anymore.

\subsection{Two parts}
\begin{enumerate}
    \item mechanism that enable the device to interrupt
    \item mechanism that handles the interrupt request.
\end{enumerate}


\subsection{Part 1}
The following must be true of the device, 
\begin{enumerate}
    \item Device wants to interrupt
    \item Device has the right to interrupt
    \item Device has higher priority than current program
\end{enumerate}

\textbf{Device wants to interrupt:} If the ready bit of the KBSR or DSR is set to 1 that means the device can be "used".

\textbf{Device has the right to interrupt:} The interrupt enable bit which can be set or cleared by the processor. Interrupt enable is bit[14] of KBSR or DSR. If 14 is set and as soon as someone types a key (15 is then set) then the device can interrupt. 

The interrupt request signal is AND of IE bit and ready bit.


\textbf{Higher priority:} LC-3 has eight priority levels. Higher the number more urgent. Priority of the device must be higher than the current program it wishes to interrupt.


\subsubsection{INT Signal}
Any device with bits [14] and [15] set assert an interrupt request signal. These are input to a priority encoder that selects the highest priority request from all those asserted. If the PL is higher than the PL of current program then INT signal is asserted.




\subsubsection{Test for INT}
We don't want to interrupt in the middle of an instruction cycle (FETCH, DECODE, EXECUTE). We'd rather have it check before it beings a new one. So before it executes a next instruction the state depends on the INT signal. So if it is not asserted it continues to FETCH else the next state is the first state of Part II.

\subsection{Part 2}
\begin{enumerate}
    \item Initiate the interrupt 
    \item Service the interrupt 
    \item Return from the interrupt
\end{enumerate}


\subsubsection{Initiate the interrupt}
\begin{enumerate}
    \item Save state of interrupted program: PC and PSR are saved. Saved on the supervisor stack the same way how TRAP does it.
    \item Load state of Interrupt Service Routine: We load PC and PSR of the interrupt service routine. Similar to TRAP service routines. Interrupt vector table is from x0100 to x01FF.
\end{enumerate}

\subsubsection{Service the interrupt}
The service routine will execute and the requirement so the device will be serviced.

\subsubsection{Return from interrupt}
RTI is executed (opcode = 1000) which consist of popping PC and PSR from supervisor stack and restoring them in the rightful places. Condition codes are also restored. If program is user privilege the user stack is restored by setting R6 to contents of Saved\_USP.


















