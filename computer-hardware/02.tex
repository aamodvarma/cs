\chapter{Data Structures}
\section{Subroutines}
We can use program fragments more efficiently. These program fragments are called subroutines/procedures or functions. So instead of writing a fragment again every time we want to use it we write it once and call it whenever we want to use it.


\textbf{JSR(R)} does two things, 

\begin{enumerate}
    \item Loads PC with starting address of the subroutine.
    \item Loads R7 with PC+1
\end{enumerate}

\textbf{JMP R7/RET} is the last instruction in the subroutine. It is used to jump back to the next instruction after the subroutine call.

\subsection{JSR(R)}
Bits [15:12] contain the opcode, 0100. Bit [11] specifies the addressing mode, 1 means PC-relative and $0$ means Base Register addressing. Lastly Bit [10:0] contain information the actual information depending on Bit[11]. For base register addressing the register is Bit[8:6].


% \subsection{Saving and Restoring Registers}






