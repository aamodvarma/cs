\chapter{Utilities}
Utilities are functions from outcomes to real numbers that describe an agent's preferences.
\begin{itemize}
    \item In a game, may be simple (+1/-1)
    \item Summarize the agent's goals
    \item Theorem: any "rational" preference can be summarized as a utility function
\end{itemize}
We hard-wire utilities and let behaviors emerge
 

\section{Maximum Expected Utility }
\begin{itemize}
    \item A rational agent should choose the action that maximize its expected utility, given its knowledge.
    \item For average case expectimax reasoning, we need magnitudes to be meaningful
\end{itemize}


\section{Preferences}
An Agent must have preferences among:
\begin{itemize}
    \item Prices: A, B, etc
    \item Lotteries: situations with uncertain prizes $L = [p,A; (1 - p),B$
\end{itemize}

\subsection{Rational Preferences}
We want some  constraints on preferences before we call them rational eg, 
$$ \text{Axiom of Transitivity: } (A > B) \wedge (B > C) \implies (A > C) $$ 

There are costs to irrationality. An agent with intransitive preferences can be induced to give away all its money

Axioms of rationality, 
\begin{enumerate}
    \item Orderability:  $(A > B) \vee (B > A) \vee (A \sim B)$
    \item Transitivity  $(A > B) \vee (B > C) \vee (A > C)$
    \item Continuity  $(A > B > C) \implies \exists p[p, A; 1 - p, C] \sim B$
    \item Substitutability  $(A \sim B) \implies [p,A; 1 - p, C] \sim [p, B, 1 - p, C]$
    \item Monotonicity  $(A > B) \implies (p \ge q) \iff [p, A; 1 - p, B ] >^{\sim} [q, A; 1 - q, B]$
\end{enumerate}


\section{Utilities of Sequences}
What preferences should an agent have over prize sequences?

More or less? $[1,2,2]$ or  $[2,3,4]$

Now or later? $[0,0,1]$ or  $[1,0,0]$ 


\subsection{Stationary preferences}
If we assume stationary preferences, 
$$ [a_1,a_2,\dots] > [b_1,b_2,\dots] \iff [c,a_1,a_2,\dots] > [c, b_1, b_2 ,\dots] $$ 

Additive discounted utility:
$$ U([r_0,r_1,r_2,\dots]) = r_0 + \gamma r_1 + \gamma^2 r_2 + \dots $$ 
where $\gamma \in (0,1]$ is the discount factor


