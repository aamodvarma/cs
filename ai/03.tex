\chapter{Adversarial Search}

We are in competitive environments where agents goals are in conflict. Most common games are \textbf{zero-sum games}  of \textbf{perfect information}. Which means its \textbf{deterministic} in a fully \textbf{observable environment}. The utility values at the end are always equal and opposite (eg. in chess one wins and other loses). Games are hard to solve. Chess games can go upto 50 moves by each player, if branching factor is around 35 that means the search tree has $35^{100}$ or $10^{154}$ nodes (search graph has $10^{40}$). Optimality and efficiency must be balanced for games.

\vspace{1em}
\textbf{Pruning}  allows us to ignore portions of the search tree that make no difference to the final choice and \textbf{heuristic evaluation functions} allow us to approximate the true utility of a state without doing complete search.

\vspace{1em}
Consider a game with two players, MIN and MAX. MAX moves first and they take turns until the game is over. At the end, points are given to the winner and taken from the loser. For instance tic-tac-toe. Initially MAX has 9 possible moves. Plays alternative between MAX and MIN. Number on each leaf node is the utility value of the terminal state from the pov of MAX; so high value are good for MAX and bad for MIN.


\section{Optimal decisions}
In adversarial search, MIN prevents us from reaching the goal state (pov of MAX). So, MAX needs to find a contingent strategy to figure out a move. So MAX must consider every possible response by MIN to its moves.
\vspace{1em}
Given a game tree, the optimal strategy can be determined from the \textbf{minimax value}  of each node. The minimax value of a node is the utility of being in the corresponding state assuming that both players play optimally from there to the end. 

\subsection{Minimax algorithm}
Computes the minimax decision from the current state. Uses recursive computation of the minimax values of each successor. Time complexity is $O^{b^{m}}$ as it does depth first search to identify value for all possible paths.

