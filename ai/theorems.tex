\chapter*{Theorems}
We start with, 

\begin{theorem}
If a list, $v_1, \dots, v_m$ spans $V$ then we can shrink it (and keep it spanning), if its not linearly independent.
\end{theorem}

\begin{theorem}
    If a list $v_1, \dots, v_m$ is linearly independent, then we can extend it (and it is still linearly independent) if the list is not spanning.
\end{theorem}


\begin{theorem}
    Length of linearly independent set $\leq$ length of spanning set
\end{theorem}

\begin{definition}
    A vector space is finite-dimentional if some list of vectors span the space
\end{definition}

\begin{definition}
    A basis is a list of vectors that are linearly independent and spanning
\end{definition}


We prove the following, 

\begin{prop}
    Every spanning list can be reduced to a basis.

\end{prop}

    \begin{proof}
        We know from theorem 1 that we can shrink every spanning list and keep it spanning, if its not linearly independent. 

        So given a spanning list, we can remove a vector and have it spanning. 

        We then have two cases, (1) the resultant list is linearly independent or (2) the list is linearly dependent.

        If the list is linearly independent, then we have a spanning list that is lienarly independent which makes it a basis.

        If the list is linearly dependent, then we can shrink it once again and keep it spanning and apply the same cases as above.

        We know that this won't go on until the list is empty because from theorem 3 we know that the length of the spanning set is lowerbounded by the length of linearly independent sets. So there will be a point where our spanning set will be linearly independent. 
    \end{proof}


\begin{prop}
    Every linearly independent list extends to a basis.
\end{prop}

    \begin{proof}
        We know from theorem 2 that we can extend a linearly independent set keeping it linearly independent if the list is not spanning.

        We given a linearly independent list, we can add a vector such that the new list is linearly indendent. We have two cases now (1) It is spanning, (2) It is not spanning.

        If it is spanning, then we have a linearly independent set of vectors that are spanning which forms a basis.

        If it is not spanning, then we can repeat this and extend our list once more keeping it linearly independent.

        Using theorem 3 we know that the length of the set of linearly independent set of vectors is upperbounded by the length of a spanning set. Hence there will come a point where our list of vectors are linearly indpendent and spanning.

        Hence they will form a basis.
    \end{proof}

\begin{theorem}
    Every vector space has a basis
\end{theorem}

    \begin{proof}
        We know by definition a finite-dimenstional vector space can be spanned by a list of vectors.

        Now from proposition 1 we know that every spanning list can be reduced to a basis.

        Hence the list spanning any finite-dimentional vector space can be reduced to a basis for the vector space.
    \end{proof}

\begin{theorem}
    Any two basis of a vector space $V$ have the same length
\end{theorem}

    \begin{proof}
        Lets assume that we can have two basis of different lengths.

        Let $a$ and $b$ be the lengths of the basis.

        We know that both the sets are linearly indpendnet and spanning. 

        Without loss of generality assume, $a < b$
        
        Using theorem 3 we also know that the length of a linearly independent list is always smaller than or equal to the length of the spanning list.

        Now, $a$ is a spanning list because it is a basis and  $b$ is a linearly independent list as it is a basis.

        So this means that the length of a spanning list is smaller than the length of a linearly independent list.

        But this contradicts theorem 3, hence our assumption must be wrong and the length of the basis must be equal.
    \end{proof}

\begin{definition}
    The length of any basis V is the dimension of V denoyed by $\dim V$
\end{definition}



\begin{prop}
    Any linearly independent list of length $\dim V$ is a basis.
\end{prop}
\begin{proof}
    Assume there exists a linearly independent list of length \dim V that is not a basis.

    That means that we are able to extend the list according to theorem 2 such that it is still linearly independent and not spanning. 

    So now we have a linearly indpendent list of vectors of length $\dim V + 1$. where  $\dim V$ is the lenght of the basis (which is a spanning set)

    However, this contradicts theorem 3 that states that the length of a linearly independent list is always smaller than or equal to the length of a spanning list. Hence our assumptino must be wrong and any liinearly independent lsit of length  $\dim V$ is a basis.
\end{proof}

\begin{prop}
    Any spanning list of length $\dim V$ is a basis.
\end{prop}
\begin{proof}
    Assume the contrary.

    Using theorem 1 we can shrink the list and keep it spanning. 

    But this contradicts theorem 3 as now we have a spanning set that is smaller than an linearly independnet list of vectors (the basis).

    So our assumption must be wrong and any spanning list of length $\dim V$ is a basis.
\end{proof}

\begin{lemma}
    Every subspace of a finite-dimentional vector space $V$ is a finite dimentional.
\end{lemma}
\begin{proof}
    Consider any vector in our subspace, $u$. As this set is linearly in dependent we can keep extending this set and keep it linearly indpendent.

    However we know that any linearly independent set is always smaller than or equal to the spanning set. 

    This set exist is linearly independent in the bigger subspace as well. So we know there exist an upperbound for our linearly indpendent set.    

    So we can keep adding a new vector, $u_k$ to our exisisting linearly indpendent list. However we know this process will terminate as there exists an upperbound which is the length of the spanning list in  $V$. Therefore  $U$ has to be finite-dimentional.
\end{proof}

\begin{lemma}
    If $U$ is a subspace of  $V$ then,  $\dim U \leq \dim V$
\end{lemma}
\begin{proof}
    Assume the contrary that $\dim U > \dim V$

    Then we have a linearly independent set of vectors that span $U$. But we know that they are linearly indpendent in $V$ as well as  $U \subseteq V$. So this means that there exists a linearly independent set of vectors in  $V$ which is larger than the length of the spanning set ($\dim V$). But this contradicts theorem 3.

    Hence our assumption must be wrong and  $\dim U \leq \dim V$
\end{proof}

\begin{lemma}
    If $U$ is a subspace of $V$ and $\dim U = \dim V$, then  $U = V$
\end{lemma}
\begin{proof}
    We can take a basis for $U$,  $u_1,u_2,\dots,u_n$  that spans $U$. Now these are also a linearly independent set of vectors in  $V$. So this means that these vectors span  $V$ as they are equal to  $\dim V$. So this means that every vector in  $V$ is a linearly combination of this span. Which implies that $U = V$

    
\end{proof}

